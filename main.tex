\documentclass[conference]{IEEEtran}
\IEEEoverridecommandlockouts
% The preceding line is only needed to identify funding in the first footnote. If that is unneeded, please comment it out.
\usepackage{cite}
\usepackage{amsmath,amssymb,amsfonts}
\usepackage{algorithmic}
\usepackage{graphicx}
\usepackage{textcomp}
\usepackage{xcolor}
\def\BibTeX{{\rm B\kern-.05em{\sc i\kern-.025em b}\kern-.08em
		T\kern-.1667em\lower.7ex\hbox{E}\kern-.125emX}}
\begin{document}
	
	\title{Classification of Fruits and Vegetables on a Conveyor}
		\author{\IEEEauthorblockN{Arvind Suresh}
		\and
		\IEEEauthorblockN{Sriram Suresh}
		\and
		\IEEEauthorblockN{Enoch Ramesh}
	}
	
	\maketitle
	
	\begin{abstract}
		This document is a model and instructions for \LaTeX.
		This and the IEEEtran.cls file define the components of your paper [title, text, heads, etc.]. *CRITICAL: Do Not Use Symbols, Special Characters, Footnotes, 
		or Math in Paper Title or Abstract.
	\end{abstract}
	
	\begin{IEEEkeywords}
		component, formatting, style, styling, insert
	\end{IEEEkeywords}
	
	\section{Introduction}
	Economic growth, increasing global per capita incomes, and rising consumer spending are all contributing to a boom in demand for a huge number of consumer products and other commodities. Fruit and vegetable industries are no exception \cite{b1_1}. However, with increasing demand comes increasing challenges for producers and processors. Some challenges are far from unique to this sector, such as growing competition \cite{b1_2} and environmental concerns \cite{b1_3}. Unlike many products though, fruits and vegetables face the imminent threat of spoilage and need to be handled and processed quickly, which puts a great burden on production lines to be quick and effective. Conveyor belts are used extensively for this reason to perform tasks such as sorting, washing, and packaging fruits and vegetables \cite{b1_4}. Machine learning (ML) can be used to classify different types of fruits and vegetables travelling on a conveyor to further aid in some aspects, such as inventory tracking and sorting.

Deep neural networks (DNNs) have advanced to encompass various areas of problems \cite{b1_5,b1_6}. However, due to the ability to have a large number of hidden layers \cite{b1_7}, DNNs can become computationally intensive and time consuming to train and run \cite{b1_8}. $\langle$ Placeholder text for methods used in this paper $\rangle$

$\langle$ Placeholder text for results $\rangle$
	
	\section{Related Literature}
	In the agriculture field, remote sensing with intelligent processing (RSIP) technologies are used for monitoring purposes \cite{b2_1}. This has also been extended to classification with the use of polarimetry synthetic aperture Radar (PolSAR). PolSAR has had plenty of successful applications in various fields in the past \cite{b2_2,b2_3,b2_4}, including crop classification and monitoring \cite{b2_5,b2_6}. PolSAR can obtain polarimetric information through the use of electromagnetic waves and has become one of the mainstreams in microwave remote sensing. For crop classification, polarimetric features, augmented with multi-temporal data, is used with a deep convolutional neural network (CNN) to achieve high accuracies with a low amount of training samples \cite{b2_7}.

Other classification techniques have been developed and used in the area of fruits and vegetables. In applications such as grading fruits and other agricultural products, features such as size, shape, colour and texture are used \cite{b2_8,b2_9,b2_10}. Accuracy is improved through the use of incorporating depth when capturing red-green-blue (RGB) images. The use of RGB-depth (RGB-D) images and CNNs have been demonstrated to be excellent for classification and recognition tasks \cite{b2_11,b2_12}, including fruits and vegetables \cite{b2_13}.

While deep CNNs are the most predominantly used method in applications of single and multiple types of vegetable classification \cite{b2_14,b2_15,b2_16}, some examples that are part of bigger systems such as farm robots \cite{b2_17} or smart fridges \cite{b2_18} have successfully employed feature extraction methods such as local binary patterns (LBP) and histogram oriented gradients (HOG) \cite{b2_19,b2_20}. In this paper, we propose augmenting the image data with LBP feature extraction to use with fast classifiers.
	
	\begin{thebibliography}{00}
		\bibitem{b1_1} ``Fruit and Vegetable Conveyor Systems from mk North America,'' mkNorthAmerica. [Online]. Available: https://www.mknorthamerica.com/Blog/fruit-and-vegetable-conveyor-systems-from-mk-north-america/.
		\bibitem{b1_2} V. Réquillart, M. Simioni, and X.L. Varela-Irimia, ``Imperfect Competition in the Fresh Fruit and Vegetable Industry'', European Association of Agricultural Economists, 113th Seminar, Sept. 2009.
		\bibitem{b1_3} ``Environmental Guidelines for Fruit and Vegetable Processing.'' [Online]. Available: https://www.miga.org/sites/default/files/ archive/Documents/FruitandVegetableProcessing.pdf.
		\bibitem{b1_4} ``Use Of Food Conveyor Belts In Food Handling Conveyors,'' Alpha Conveyor. [Online]. Available: https://www.alphaconveyor.com/products/food-grade-conveyor/use-of-food-conveyor-belts-in-food-handling-conveyors/.
		\bibitem{b1_5} A. Ren, Z. Li, C. Ding, Q. Qiu, Y. Wang, J. Li, X. Qian, and B. Yuan, ``Sc-dcnn: Highly-scalable deep convolutional neural network using stochastic computing'', Proceedings of the Twenty-Second International Conference on Architectural Support for Programming Languages and Operating Systems, pp. 405-418, 2017.
		\bibitem{b1_6} Z. Li, A. Ren, J. Li, Q. Qiu, B. Yuan, J. Draper, and Y. Wang, ``Structural design optimization for deep convolutional neural networks using stochastic computing'', 2017 Design Automation \& Test in Europe Conference \& Exhibition (DATE), pp. 250-253, 2017.
		\bibitem{b1_7} K. Simonyan and A. Zisserman, ``Very deep convolutional networks for large-scale image recognition,'' 2014.
		\bibitem{b1_8} G.-B. Huang, Q.-Y. Zhu, and C.-K. Siew,``Extreme learning machine: theory and applications'', Neurocomputing, vol. 70, no. 1, pp. 489-501, 2006.
		\bibitem{b2_1} Yu Haiyang, Liu Yanmei, Yang Guijun, and Yang Xiaodong, ``Quick image processing method of HJ satellites applied in agriculture monitoring,'' in 2016 World Automation Congress (WAC), 2016.
		\bibitem{b2_2} M. Sato, S.-W. Chen, and M. Satake, ``Polarimetric SAR analysis of tsunami damage following the March 11 2011 East Japan Earthquake'', Proc. IEEE, vol. 100, no. 10, pp. 2861-2875, Oct, 2012.
		\bibitem{b2_3} S.-W. Chen and M. Sato, ``Tsunami damage investigation of built-up areas using multitemporal spaceborne full polarimetric SAR image'', IEEE Trans. Geosci. Remote Sens., vol. 51, no. 4, pp. 1985-1997, Apr. 2013.
		\bibitem{b2_4} S. W. Chen, X. S. Wang, and M. Sato, ``Urban damage level mapping based on scattering mechanism investigation using fully polarimetric SAR data for the 3.11 East Japan earthquake'', IEEE Trans. Geosci. Remote Sens., vol. 54, no. 12, pp. 6919-6929, 2016.
		\bibitem{b2_5} J. M. Lopez-Sanchez, F. Vicente-Guijalba, J. D. Ballester-Berman, and S. R. Cloude, ``Polarimetric response of rice fields at C-band: Analysis and phenology retrieva'', IEEE Trans. Geosci. Remote Sens., vol. 52, no. 5, pp. 2977-2993, May 2014.
		\bibitem{b2_6} C. Yonezawa, M. Negishi, K. Azuma, M. Watanabe, N. Ishitsuka, S. Ogawa, and G. Saito, ``Growth monitoring and classification of rice fields using multitemporal RADARSAT-2 full-polarimetric data'', Int. J. Remote Sens., vol. 33, no. 18, pp. 5696-5711, 2012.
		\bibitem{b2_7} S.-W. Chen and C.-S. Tao, ``Multi-temporal PolSAR crops classification using polarimetric-feature-driven deep convolutional neural network,'' in 2017 International Workshop on Remote Sensing with Intelligent Processing (RSIP), 2017.
		\bibitem{b2_8} J. Gill, P.S. Sandhu, and T. Singh, ``A Review of Automatic Fruit Classification using Soft Computing Techniques'', International Conference on Computer Systems and Electronics Engineering (ICSCEE'2014), pp. 91-98, 2014.
		\bibitem{b2_9} M. Raj and D. Swaminarayan, ``Applications of Image Processing for Grading Agriculture products'', International Journal on Recent and Innovation Trends in Computing and Communication, vol. 3, no. 3, pp. 1194-1201, 2015.
		\bibitem{b2_10} S. Banot and P.M. Mahajan, ``A Fruit Detecting and Grading System Based on Image Processing-Review'', International Journal of Innovative Research in Electrical Electoronics Instrumentation and Control Engineering, vol. 4, no. 1, 2016.
		\bibitem{b2_11} R. Socher, B. Huval, B. Bhat, C.D. Manning, and A.Y. Ng, ``Convolutional-Recursive Deep Learning for 3D Object Classification'', Advances in Neural Information Processing Systems 25 (NIPS 2012), 2012.
		\bibitem{b2_12} A. Eitel, J.T. Springenberg, L. Spinello, M. Riedmiller, and W. Burgard, ``Multimodal Deep Learning for Robust RGB-D Object Recognition'', IEEE/RSJ International Conference on Intelligent Robots and Systems (IROS), 2015.
		\bibitem{b2_13} T. Nishi, S. Kurogi, and K. Matsuo, ``Grading fruits and vegetables using RGB-D images and convolutional neural network,'' in 2017 IEEE Symposium Series on Computational Intelligence (SSCI), 2017.
		\bibitem{b2_14} ``How a Japanese cucumber farmer is using deep learning and TensorFlow,'' Google Cloud Blog. [Online]. Available: https://cloud.google.com/blog/products/gcp/how-a-japanese-cucumber-farmer-is-using-deep-learning-and-tensorflow.
		\bibitem{b2_15} Y. Sakai, T. Oda, M. Ikeda, and L. Barolli, ``A Vegetable Category Recognition System Using Deep Neural Network,'' in 2016 10th International Conference on Innovative Mobile and Internet Services in Ubiquitous Computing (IMIS), 2016.
		\bibitem{b2_16} O. Patil, ``Classification of Vegetables using TensorFlow,'' International Journal for Research in Applied Science and Engineering Technology, vol. 6, no. 4, pp. 2926–2934, Apr. 2018.
		\bibitem{b2_17} A. B. Titus, T. Narayanan, and G. P. Das, ``Vision system for coconut farm cable robot,'' in 2017 IEEE International Conference on Smart Technologies and Management for Computing, Communication, Controls, Energy and Materials (ICSTM), 2017.
		\bibitem{b2_18} Shweta A.S, ``Intelligent refrigerator using ARTIFICIAL INTELLIGENCE,'' in 2017 11th International Conference on Intelligent Systems and Control (ISCO), 2017.
		\bibitem{b2_19} H.-L. Kuang, L. L. H. Chan, and H. Yan, ``Multi-class fruit detection based on multiple color channels,'' in 2015 International Conference on Wavelet Analysis and Pattern Recognition (ICWAPR), 2015.
		\bibitem{b2_20} N. Kulu, M. Baskaya, A. Keles, A. Altan, and R. Hacioglu, ``Determination of Fruit Health Status and Yield with Unmanned Aerial Vehicle,'' in 2018 2nd International Symposium on Multidisciplinary Studies and Innovative Technologies (ISMSIT), 2018.
	\end{thebibliography}
\end{document}
