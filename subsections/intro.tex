Economic growth, increasing global per capita incomes, and rising consumer spending are all contributing to a boom in demand for a huge number of consumer products and other commodities. Fruit and vegetable industries are no exception \cite{b1_1}. However, with increasing demand comes increasing challenges for producers and processors. Some challenges are far from unique to this sector, such as growing competition \cite{b1_2} and environmental concerns \cite{b1_3}. Unlike many products though, fruits and vegetables face the imminent threat of spoilage and need to be handled and processed quickly, which puts a great burden on production lines to be quick and effective. Conveyor belts are used extensively for this reason to perform tasks such as sorting, washing, and packaging fruits and vegetables \cite{b1_4}. Machine learning (ML) can be used to classify different types of fruits and vegetables travelling on a conveyor to further aid in some aspects, such as inventory tracking and sorting.

Deep neural networks (DNNs) have advanced to encompass various areas of problems \cite{b1_5,b1_6}. However, due to the ability to have a large number of hidden layers \cite{b1_7}, DNNs can become computationally intensive and time consuming to train and run \cite{b1_8}. This motivated us to look for a less resource intensive method that can be trained with a smaller sample size for this application. We propose to preprocess fruit and vegetable images and augment feature extractions from them which are then used with an ensemble classifier to quickly classify the fruit or vegetable. For this method, we assume that the fruit and vegetable pictures are cropped from the whole conveyor using various image processing techniques \cite{b1_9}.