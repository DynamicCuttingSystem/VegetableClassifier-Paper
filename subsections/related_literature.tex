In the agriculture field, remote sensing with intelligent processing (RSIP) technologies are used for monitoring purposes \cite{b2_1}. This has also been extended to classification with the use of polarimetry synthetic aperture radar (PolSAR). PolSAR has had plenty of successful applications in various fields in the past \cite{b2_2,b2_3,b2_4}, including crop classification and monitoring \cite{b2_5,b2_6}. PolSAR can obtain polarimetric information through the use of electromagnetic waves and has become one of the mainstreams in microwave remote sensing. For crop classification, polarimetric features, augmented with multi-temporal data, is used with a deep convolutional neural network (CNN) to achieve high accuracies with a low amount of training samples \cite{b2_7}.

Other classification techniques have been developed and used in the area of fruits and vegetables. In applications such as grading fruits and other agricultural products, features such as size, shape, colour and texture are used \cite{b2_8,b2_9,b2_10}. Accuracy is improved through the use of incorporating depth when capturing red-green-blue (RGB) images. The use of RGB-depth (RGB-D) images and CNNs have been demonstrated to be excellent for classification and recognition tasks \cite{b2_11,b2_12}, including fruits and vegetables \cite{b2_13}.

While deep CNNs are the most predominantly used method in applications of single and multiple types of vegetable classification \cite{b2_14,b2_15,b2_16}, some examples that are part of bigger systems such as farm robots \cite{b2_17} or smart fridges \cite{b2_18} have successfully employed feature extraction methods such as local binary patterns (LBP) and histogram oriented gradients (HOG) \cite{b2_19,b2_20}. In this paper, we propose augmenting the image data with LBP and HOG feature extraction to use with fast ensemble classifiers.